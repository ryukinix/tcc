\documentclass{templatebeamerufc/libs/ufc_format}
% Insere preâmbulo com a importação de pacotes
\input{templatebeamerufc/libs/preamble.tex}
% Macros pessoais
%%% Local Variables:
%%% mode: latex
%%% TeX-master: t
%%% End:

% macros de utilidade
\newcommand{\fonteautor}{Elaborado pelo autor.}
\newcommand{\fonteautorscikit}{Elaborado pelo autor através de imagens
disponíveis na biblioteca \textit{scikit-image}~\cite{scikit-image}.}
\newcommand{\red}[1]{\textcolor{red}{\textbf{#1}}}  % bold red text

\listfiles
% Título
\title[EGSIS]{\textbf{Segmentação Semi-Supervisionada de Imagens através de
Dinâmicas Coletivas em Redes Complexas}}
% Subtítulo
\subtitle{Uma avaliação no cenário de segmentação interativa}
% Autor da apresentação
\author{Manoel Vilela Machado Neto}
% Nome do instituto
\institute[UFC]{
    % email para contato
    \normalsize{\email{manoel.machado@alu.ufc.br}}
    \newline
    % Nome do departamento
    \department{Engenharia da Computação}
    \newline
    % Nome da universidade
    \ufc{}
}
% date of the presentation
\date{Sobral, \today}


%%%%%%%%%%%%%%%%%%%%%%%%%%%%%%%%%%%%%%%%%%%%%%%%%%%%%%%%%%%%%%%%%%%%%%%%%%%%%%%%%%
%% Inicío do Documento de Apresentação                                          %%
%%%%%%%%%%%%%%%%%%%%%%%%%%%%%%%%%%%%%%%%%%%%%%%%%%%%%%%%%%%%%%%%%%%%%%%%%%%%%%%%%%
\begin{document}
% Insere o estílo de código (quase não usado)
\input{templatebeamerufc/libs/code_style}
%% ---------------------------------------------------------------------------
% Primeiro slide (com título, subtítulo, ...)
\begin{frame}{}
    \maketitle
\end{frame}

%% ---------------------------------------------------------------------------
% Segundo slide com sumário
\begin{frame}[allowframebreaks]{Sumário}
  \tableofcontents[sections={1-2}]
    \framebreak{}
  \tableofcontents[sections={3-5}]
\end{frame}

%% ---------------------------------------------------------------------------------
%% ------------ INTRODUÇÃO -------------------------------
%% ---------------------------------------------------------------------------------
\section{Introdução}

%% ---------------------------------------------------------------------------
\subsection{Tipos de segmentação de imagem}

\begin{frame}{Segmentação de imagens}
  \begin{block}{Definição}
    Segmentar uma imagem significa dividí-la em regiões de interesse.
  \end{block}

  Existem alguns tipos diferentes de segmentação de imagens, como por
exemplo:
  \begin{enumerate}
    \item Segmentação semântica;
    \item Segmentação de instâncias;
    \item Segmentação interativa.
  \end{enumerate}
\end{frame}


\begin{frame}{Segmentação de instâncias e semântica}
  % imagem comparando segmentação semântica vs segmentação por instâncias
  \begin{figure}\label{fig:semantic-vs-instance-segmentation}
    \centering
    \caption{Comparação entre segmentação semântica e segmentação de instâncias.}
    \includegraphics[scale=0.23]{figuras/image-segmentation-types}
    \source{\cite{MediumInstanceSegmentation2019}}
  \end{figure}
\end{frame}

\begin{frame}{Segmentação interativa}
  \begin{figure}\label{fig:interactive--segmentation}
    \centering
    \caption{Ilustração de segmentação interativa, rotulações em azul  e
      vermelho.\\ Na imagem à direita, após segmentação o fundo foi removido.}
    \includegraphics[scale=0.6]{figuras/interactive-segmentation-2008}
    \source{\cite{duchenne2008segmentation}}
  \end{figure}
\end{frame}

\begin{frame}{Segmentação interativa}
  \begin{alertblock}{Foco}
  O foco desta apresentação é demonstrar uma \textbf{nova técnica de
  segmentação de imagens}, avaliada no cenário de segmentação
  interativa com \textbf{duas classes}: o objeto de interesse e o plano de fundo.
  \end{alertblock}
\end{frame}


%% ---------------------------------------------------------------------------
\subsection{Aprendizado de máquina}

\begin{frame}{Aprendizado de máquina}
  Em~\cite{samuel1959some}, aprendizado de maquina é definido como:

  \begin{block}{Definição} Campo de estudo que dá aos computadores a
    habilidade de aprender sem serem explicitamente programados.
  \end{block}

  \pause{}

  \begin{figure}\label{fig:samuel}
    \centering
    \caption{
      Arthur Samuel, 1956, apresenta sua criação ao público na
      TV, uma inteligência artificial capaz de jogar damas no
      computador IBM 701.
    }
    \includegraphics[scale=0.18]{figuras/samuel}
    \source{\cite{press2021machine}}
  \end{figure}
\end{frame}

\begin{frame}{Aprendizado semi-supervisionado}
  \begin{figure}\label{fig:ilustraca;ao-aprendizado-semi-supervisionado}
    \centering
    \caption{
      O aprendizado semi-supervisionado tem como principal
característica \\ a utilização de dados rotulados e não rotulados na base
de treinamento.
}
    \includegraphics[scale=0.1]{figuras/ilustracao-aprendizado-semi-supervisionado}
    \source{\fonteautor}
  \end{figure}
\end{frame}


\begin{frame}{Aprendizado semi-supervisionado}
  \begin{block}{Dados de treinamento}
    Dado um conjunto de dados de treinamento $ \mathbf{X} =
    \{\vec{x_1}, \vec{x_2}, \ldots, \vec{x_n}\} $, tal que $ \vec{x_n} \in \mathbb{R}^d $,
    apenas um subconjunto $ \vec{Y} = \{y_1, y_2, \ldots , y_m\} $,
    em que $ (m < n) $, tem rótulos correspondentes.
  \end{block}

  \pause{}

  \begin{block}{Objetivo}
    O objetivo do aprendizado semi-supervisionado é usar tanto o
conjunto de dados rotulado quanto o não rotulado para aprender a
função $ f: \mathbf{X} \rightarrow \vec{Y} $ que pode prever o rótulo $ y $ para
um novo exemplo $ \vec{x} $.
  \end{block}
\end{frame}


\subsection{Aprendizado indutivo vs aprendizado transdutivo}

\begin{frame}{Aprendizado indutivo vs aprendizado transdutivo}
  \begin{figure}\label{fig:induction-vs-transudciton-cropped}
    \centering
    \caption{
      No aprendizado indutivo, como em redes neurais, uma função de
inferência é estimada durante o treinamento. Enquanto isso,
no aprendizado transdutivo a inferência de novos pontos é realizado sem a
necessidade de estimar essa função.
}
    \includegraphics[scale=0.42]{figuras/induction_vs_transduction_cropped}
    \source{\cite{vapnik1995}}
  \end{figure}
\end{frame}

\begin{frame}{Aprendizado indutivo vs aprendizado transdutivo}
  \begin{figure}\label{fig:transductive-vs-inductive}
    \centering
    \caption{ Uma ilustração de aprendizado indutivo e aprendizado
      transdutivo.  }
    \includegraphics[scale=0.75]{figuras/transductive-vs-inductive}
    \source{\cite{vapnik1995}}
  \end{figure}
\end{frame}

\begin{frame}{Aprendizado indutivo vs aprendizado transdutivo}
  \begin{exampleblock}{Analogia}
     Através de uma analogia, pode-se considerar o aprendizado
indutivo como um sistema de educação que alcança o entendimento generalizado,
enquanto o aprendizado transdutivo é um sistema de educação focado
em realização de provas.
  \end{exampleblock}
\end{frame}


%% -----------------------------------------------------------------------------------
%% ----- FUNDAMENTAÇÃO TEÓRICA ---------
%% ------------------------------------------------------------------------------------
\section{Fundamentação Teórica}
\subsection{Superpixels}
\subsection{Redes complexas}
\subsection{Extração de características}
\subsection{Dinâmicas coletivas}
\subsection{EGSIS}

%% -----------------------------------------------------------------------------------
%% ---------- METODOLOGIA ------------------------------------
%% -------------------------------------------------------------------------------------
\section{Metodologia}

%%
\subsection{Dataset GrabCut}

\subsection{Métricas de avaliação}


\section{Resultados}

\subsection{Variação na quantidade de superpixels}
\subsection{Resultados qualitativos}
\subsection{Resultados quantitativos}

\section{Conclusão}
\begin{frame}{Principais conclusões}
  \begin{alertblock}{WIP}
    Ler documento de conclusao e sumarizar
  \end{alertblock}
\end{frame}

\subsection{Limitações e desafios}
\subsection{Trabalhos futuros}


%% ---------------------------------------------------------------------------
% Slides de referência
\begin{frame}[allowframebreaks]
  \frametitle{Referências}
  % Inserting the references file
  \bibliography{3-pos-textuais/referencias.bib}
  %\printbibliography{}
\end{frame}

%% ---------------------------------------------------------------------------
% Slide final com agradecimento e contato
\begin{frame}{}
    \centering
    \huge{\textbf{\example{Obrigado pela atenção!}}}

    \vspace{1cm}

    \Large{\textbf{Contato:}}
    \newline
    \vspace*{0.5cm}
    \large{\email{manoel.machado@alu.ufc.br}}
\end{frame}

\end{document}
