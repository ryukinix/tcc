%%%%%%%%%%%%%%%%%%%%%%%%%%%%%%%%%%%%%%%%%%%%%%%%%%%%%%%%%%%%%%%%%%%%%%%%%%%%%%%%%%%
%% This project aims to create the UFC template for presentation.                %%
%% author: Maurício Moreira Neto - Doctoral student in Computer Science (MDCC)   %%
%% contacts:                                                                     %%
%%    e-mail: maumneto@ufc.br                                                    %%
%%    linktree: https://linktr.ee/maumneto                                       %%
%%%%%%%%%%%%%%%%%%%%%%%%%%%%%%%%%%%%%%%%%%%%%%%%%%%%%%%%%%%%%%%%%%%%%%%%%%%%%%%%%%%
\documentclass{templatebeamerufc/libs/ufc_format}
% Inserting the preamble file with the packages
\input{templatebeamerufc/libs/preamble.tex}
% Inserting the references file
\bibliography{templatebeamerufc/references.bib}
\listfiles
% Title
\title[EGSIS]{\textbf{Segmentação Semi-Supervisionada de Imagens através de
Dinâmicas Coletivas em Redes Complexas}}
% Subtitle
\subtitle{}
% Author of the presentation
\author{Manoel Vilela Machado Neto}
% Institute's Name
\institute[UFC]{
    % email for contact
    \normalsize{\email{manoel.machado@alu.ufc.br}}
    \newline
    % Department Name
    \department{Engenharia da Computação}
    \newline
    % university name
    \ufc{}
}
% date of the presentation
\date{\today}


%%%%%%%%%%%%%%%%%%%%%%%%%%%%%%%%%%%%%%%%%%%%%%%%%%%%%%%%%%%%%%%%%%%%%%%%%%%%%%%%%%
%% Start Document of the Presentation                                           %%
%%%%%%%%%%%%%%%%%%%%%%%%%%%%%%%%%%%%%%%%%%%%%%%%%%%%%%%%%%%%%%%%%%%%%%%%%%%%%%%%%%
\begin{document}
% insert the code style
\input{templatebeamerufc/libs/code_style}

%% ---------------------------------------------------------------------------
% First frame (with tile, subtitle, ...)
\begin{frame}{}
    \maketitle
\end{frame}

%% ---------------------------------------------------------------------------
% Second frame
\begin{frame}{Sumário}
    \begin{multicols}{2}
        \tableofcontents
    \end{multicols}
\end{frame}

%% ---------------------------------------------------------------------------
% This presentation is separated by sections and subsections
\section{Seção I}
\begin{frame}{Explicações}
    % itemize
    Este é um template que pode ser utilizado para:
    \begin{itemize}
        \item Apresentação de Trabalhos Acadêmicos
        \item Apresentação de Disciplinas
        \item Apresentações de Teses e Dissertações
    \end{itemize}

    \vspace{0.4cm} % vertical space

    % enumeration
    Para utilizar este template corretamente é importante que:
    \begin{enumerate}
        \item Tenha conhecimento mínimo sobre LaTeX
        \item Ler os comentários no template (explicações)
        \item Ler o README.md (documentação)
    \end{enumerate}

    \vspace{0.2cm}

    \example{Este é um texto de exemplo!} \emph{Texto de Ênfase!}
\end{frame}

%% ---------------------------------------------------------------------------
\subsection{Subseção I}
\begin{frame}{Criando Blocos}
    % Blocks styles
    \begin{block}{Bloco Padrão}
        Texto do corpo do bloco.
    \end{block}

    \begin{alertblock}{Bloco de Alerta}
        Texto do corpo do bloco.
    \end{alertblock}

    \begin{exampleblock}{Bloco de Exemplo}
        Texto do corpo do bloco.
    \end{exampleblock}
\end{frame}

%% ---------------------------------------------------------------------------
\subsection{Subseção II}
\begin{frame}{Criando Caixas}
    \successbox{testando o success box}

    \pause

    \alertbox{testando o alert box}

    \pause

    \simplebox{testando o simple box}
\end{frame}

%% ---------------------------------------------------------------------------
\subsection{Subseção III}
\begin{frame}{Criando Algoritmos (Pseudocódigo)}
    \begin{algorithm}[H]
        \SetAlgoLined
        \LinesNumbered
        \SetKwInOut{Input}{input}
        \SetKwInOut{Output}{output}
        \Input{x: float, y: float}
        \Output{r: float}
        \While{True}{
          r = x + y\;
          \eIf{r >= 30}{
           ``O valor de $r$ é maior ou iqual a 10.''\;
           break\;
           }{
           ``O valor de $r$ = '', r\;
          }
         }
         \caption{Algorithm Example}
    \end{algorithm}
\end{frame}

%% ---------------------------------------------------------------------------

\begin{frame}{Inserindo Algoritmos}
    \lstset{language=Python}
    \lstinputlisting[language=Python]{templatebeamerufc/code/main.py}
\end{frame}

%% ---------------------------------------------------------------------------
\begin{frame}{Inserindo Algoritmos}
    \lstinputlisting[language=C]{templatebeamerufc/code/source.c}
\end{frame}

%% ---------------------------------------------------------------------------
\begin{frame}{Inserindo Algoritmos}
    \lstinputlisting[language=Java]{templatebeamerufc/code/helloworld.java}
\end{frame}

%% ---------------------------------------------------------------------------
\begin{frame}{Inserindo Algoritmos}
    \lstinputlisting[language=HTML]{templatebeamerufc/code/index.html}
\end{frame}

%% ---------------------------------------------------------------------------
% This frame show an example to insert multicolumns
\section{Multicolunas}
\begin{frame}{Seção II - Multicolunas}
    \begin{columns}{}
        \begin{column}{0.5\textwidth}
            \justify
            É possível colocar mais de uma coluna utilizando os comandos de $\backslash$begin\{column\}\{\} e $\backslash$end\{column\}
        \end{column}
        \begin{column}{0.5\textwidth}
            \justify
            Porém, o espaçamento deve ser proporcional entre as colunas para que estas colunas não entrem em coflito. O espaçamento é dado pelo segundo argumento do $\backslash$begin.
        \end{column}
    \end{columns}
\end{frame}

%% ---------------------------------------------------------------------------
% This frame show an example to insert figures
\section{Imagens}
\begin{frame}{Seção III - Figures}
    \begin{figure}
        \centering
        \caption{Emblema da UFC.}
        \includegraphics[scale=0.3]{templatebeamerufc/libs/emblemufc.pdf}
        \source{Obtido pelo site oficial da UFC~\cite{siteufc}~\cite{einstein}}
        \label{fig:ufc_emblem}
    \end{figure}
\end{frame}

%% ---------------------------------------------------------------------------
% Reference frames
\begin{frame}[allowframebreaks]
    \frametitle{Referências}
    \printbibliography{}
\end{frame}

%% ---------------------------------------------------------------------------
% Final frame
\begin{frame}{}
    \centering
    \huge{\textbf{\example{Obrigado pela atenção!}}}

    \vspace{1cm}

    \Large{\textbf{Contato:}}
    \newline
    \vspace*{0.5cm}
    \large{\email{manoel.machado@alu.ufc.br}}
\end{frame}

\end{document}
