\chapter{CONCLUSÕES E TRABALHOS FUTUROS}\label{chap:conclusoes-e-trabalhos-futuros}

Neste trabalho, foi possível conceber um novo algoritmo chamado de
\gls{EGSIS} que pode ser enquadrado na resolução de problemas de
segmentação interativa de imagens. Ao realizar a pesquisa sobre esse
tema, foi observado que a linha de pesquisa sobre algoritmos
transdutivos de segmentação de imagens ainda está evoluindo, mas com
menos volume que as técnicas indutivas e, por esse motivo, este
trabalho pode se posicionar como uma referência para construção de
mais trabalhos envolvendo técnicas similares.

Na seção de resultados, o algoritmo \gls{EGSIS} obteve uma performance
próxima daquelas obtidas pelos métodos estado-da-arte baseados em
grafos para segmentação interativa de imagens. Durante a exploração
dos resultados, foi observado que a qualidade da segmentação final
depende crucialmente da qualidade da pré-segmentação com superpixels.

Durante os experimentos, foi observado também que os melhores
resultados foram alcançados com um maior número de superpixels e que
o aumento do número de superpixels aumentou também o tempo de
execução do algoritmo.

O algoritmo possui várias partes móveis e trocáveis, o que permite no
futuro a evolução de técnicas para melhorar ainda mais a qualidade de
segmentação final. Os métodos de superpixel, extração de
características, construção da rede complexa e dinâmica coletiva são
partes móveis que podem ser testadas com técnicas mais avançadas.

\section{Limitações e desafios}\label{sec:limitacoes-desafios}

Um ponto de dificuldade em toda a pesquisa foi encontrar trabalhos de
referência nessas condições com a palavra-chave semi-supervisionado,
uma vez que as buscas realizadas direcionavam para trabalhos nos quais
os algoritmos eram indutivos. Após trocar para segmentação transdutiva
e segmentação interativa (mais recente), foi possível encontrar
trabalhos que poderiam se enquadrar melhor com a proposta deste trabalho.

A falta de \textit{datasets} para segmentação interativa com a
inclusão da anotação parcial inicial foi um dos maiores desafios
deste trabalho, o que limitou a capacidade de exploração e comparação
com outros datasets de segmentação de imagens, assim como outras
técnicas propostas em outros artigos. Por conta dessa ausência, o
trabalho se limitou ao dataset GrabCut.

A segmentação interativa é um tipo de problema que, por natureza, é
uma segmentação binária. O algoritmo \gls{EGSIS} permite a segmentação
interativa, mas não é limitado a esta. O desenvolvimento da técnica
considerou também segmentação multi-classes, mas, por conta da
dificuldade de encontrar \textit{datasets} públicos que contivessem as
rotulações parciais, se tornou uma limitação da pesquisa.

A exploração na linha de pesquisa sobre algoritmos de segmentação
transdutiva multi-classes ainda possui espaço para crescer. A criação
de um dataset novo para benchmark que tivesse essas anotações do
usuário, como existe no GrabCut, poderia ser uma contribuição
relevante para essa linha de pesquisa.


\section{Trabalhos futuros}\label{sec:trabalhos-futuros}

Como mencionado anteriormente, o algoritmo \gls{EGSIS} possui uma
flexibilidade na troca de várias partes que o compõem. Como potenciais
trabalhos futuros, uma das principais oportunidades de pesquisa seria
testar técnicas estado-da-arte de extração de características de
imagens. Neste trabalho, a extração de características foi limitada a
matriz de co-ocorrência e filtros de Gabor, que não são as melhores
técnicas atualmente, apesar de já terem sido suficientes para prover
bons resultados.

A segmentação interativa evoluiu tanto para melhorar a qualidade de
segmentação quanto para para reduzir ao mínimo a necessidade de
rotulações iniciais providas pelo usuário. Em publicações mais
recentes sobre segmentação interativa, novos estudos buscam criar
técnicas que minimizam o número de cliques de anotação que o usuário
precisa fazer para alcançar segmentações de boa qualidade. Um trabalho
futuro seria expandir as atuais métricas para simular a quantidade
mínima de anotação necessária para alcançar um IoU com um determinado
valor mínimo, geralmente entre 0.85 e 0.95.
