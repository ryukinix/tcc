\chapter{Revisão Bibliográfica}\label{cap:revisao-bibliografica}

Técnicas de segmentação de imagens com o paradigma semi-supervisionado
está em foco atualmente no campo médico, como pode ser visto em
\cite{LuoSemiSupervised2021}. Uma das grandes motivações dos autores
do artigo citado para utilizar uma técnica semi-supervisionada está
relacionado com a dificuldade de anotação de dados no ambiente
hospitalar, requisito para utilizar técnicas mais robustas como Mask R-CNN.

Em relação ao tópico de redes complexas e dinâmicas coletivas, é
possível mencionar o trabalho feito com o algoritmo \textit{Labeled
Component Unfolding} (LCU) \cite{VerriNetworkUnfoldingMap2018} na
qual os principais conceitos sobre resolução de problemas de
aprendizagem de máquina semi-supervisionado são explorados em detalhes
como uma dinâmica de competição de partículas na relação
vértice-arestas de um grafo não-direcionado. Este método de
aprendizagem é transdutivo, portanto, difere dos métodos indutivos
como é o caso de Redes Neurais. Além disso este algoritmo tem
complexidade computacional linear em relação a quantidade de arestas e
vértices.

Para ilustrar uma possível ideia de segmentação de imagens, ao
considerar um algoritmo que faça uma transformação do domínio de imagem para um
grafo, é possível estabelecer uma relação onde os vértices representam
parte da imagem como um \textit{Superpixel}
\cite{SuperpixelSurvey2020}, ou seja, um grupo de subpixels da
imagem. Algoritmos de Superpixel são não-supervisionados em geral,
portanto possui suas limitações quanto ao resultado esperado pelo
usuário \textendash logo difícil de ser aceito em aplicações médicas onde a
opinião do especialista é de alta relevância para o resultado final.

\begin{figure}[!h]
        \captionsetup{width=8cm}
		\Caption{\label{fig:segmentation-superpixel}
          Segmentação superpixel}
		\centering
		\UFCfig{}{
			\fbox{\includegraphics[width=8cm]{figuras/example-superpixel-segmentation}}
		}{
			\Fonte{\cite{SuperPixelBenchmark2017}}
		}
\end{figure}



Seguindo essa perspectiva, ao utilizar um algoritmo de extração de
\textit{features} de imagens sobre o Superpixel, tem-se que o vértice
do grafo é neste momento um vetor de características. O sistema de
competição proposto no artigo \textit{Network Unfolding Map By
Vertex-Edge Dynamics Modeling} pode otimizar o pertencimento de
classes (segmentos nesse caso) baseado na topologia de sua vizinhança
e a relação aos vértices conectados. A métrica de semelhança pode ser
ajustada de acordo com o problema, entre distância euclidiana,
\textit{cosine-distance} além de outras possíveis métricas de
semelhança pra competição.

Ao considerar o problema como semi-supervisionado, a pista de ter
alguns dos vértices anotados, adicionaria um \textit{bias}
parametrizado pelo conhecimento do especialista em uso da ferramenta,
como um editor ou um médico. A otimização do pertencimento das classes
então seria acionada pela dinâmica coletiva selecionada em questão,
que por acaso, poderia ser LCU mencionado anteriormente.

Por outro lado, essas técnicas por serem novas há ainda trabalhos a
serem feitos, como por exemplo: analisar as condições de convergência
do algoritmo. Isso pode ser um dos resultados desse trabalho,
demandando uma análise matemática com auxílio de experimentos empíricos.

É de importância mencionar que já foi demonstrado em outras situações,
como em \cite{JarbasComplexNetworks2020} que o uso de redes complexas
em fusão com redes neurais aleatórias podem gerar um discriminante de
textura da imagem de alta relevância como extrator de
características. Neste caso, é possível se apoiar nesse resultado como
uma evidência que a investigação de novas técnicas considerando a
topologia da imagem através de redes complexas é uma oportunidade de
pesquisa.
