\chapter{METODOLOGIA}\label{cap:metodologia}

Para avaliação deste trabalho, foi considerada a base de dados
GrabCut~\cite{rother2004grabcut}, que contém 50 imagens com
segmentação binária e anotações parciais para execução de segmentações
assistidas, assim como segmentação interativa. Nessa
coleção de imagens, há variados tipos de objetos, como pessoas,
carros, plantas, animais, entre outros tipos de imagem. Na Figura~\ref{fig:imagens-avaliadas} é
apresentado uma amostra de 16 imagens que serão usadas para realizar
uma avaliação qualitativa do método EGSIS.\@

\begin{figure}[h!]
        \captionsetup{width=16cm}
		\Caption{\label{fig:imagens-avaliadas}
          Amostra de 16 imagens do \textit{dataset} GrabCut.
        }
		\centering
		\UFCfig{}{\fbox{\includegraphics[width=16cm]{figuras/imagens-avaliadas}}}
        {\Fonte{Elaborado pelo autor baseado em~\citeonline{rother2004grabcut}}}
\end{figure}
\FloatBarrier{}

As anotações parciais que serão consideradas neste trabalho são
chamadas de \textit{Lasso} e simulam um usuário realizando uma
marcação de regiões do plano de fundo e primeiro plano para
segmentação. A anotação possui um esquema de anotação para treinamento
em escala de cinza e será descrita em sequência.

\begin{figure}[h!]
        \captionsetup{width=12cm}
		\Caption{\label{fig:grabcut-dataset}
          Exemplo de imagem \textit{dataset} GrabCut com anotação parcial e
          a segmentação real (\textit{ground truth}).
        }
		\centering
		\UFCfig{}{\fbox{\includegraphics[width=12cm]{figuras/grabcut-dataset}}}
        {\Fonte{Elaborado pelo autor baseado em~\citeonline{rother2004grabcut}}}
\end{figure}
\FloatBarrier{}

\begin{table}[h]
  \centering
  \caption{Regras da anotação Lasso do GrabCut}\label{tab:grabcut-label}
  \begin{tabular}{lc}
    \toprule
    Descrição                                       & Nível de cinza \\
    \midrule \midrule
     fundo                                          & 0              \\
     fundo usado para \textbf{treinamento}          & 64             \\
     região de inferência                           & 128            \\
     primeiro plano usado para \textbf{treinamento} & 255            \\
    \bottomrule
  \end{tabular}
  \Fonte{\cite{rother2004grabcut}}
\end{table}


Através da Figura~\ref{fig:grabcut-dataset}, ao observar-se a primeira
imagem na segunda linha, tem-se a anotação \textit{Lasso}, na qual o
significado dos níveis de cinza pode ser visualizado na
Tabela~\ref{tab:grabcut-label}. Dessa maneira, para
treinamento\footnotemark{} são usados dois níveis de cinza: 64 para o
fundo e 255 para o primeiro plano.

\footnotetext{\gls{EGSIS} é um algoritmo transdutivo,
não existe treinamento, esses dados são apenas anotações parciais
que serão usadas na execução da inferência para obter a segmentação da
imagem.}

O método de avaliação consiste em executar o algoritmo desenvolvido
\gls{EGSIS} para essa base de dados e extrair métricas de qualidade de
segmentação para comparação com outros trabalhos, assim como estudo da
variação de parâmetros do modelo e a percepção de impacto nas métricas
de avaliação.

\section{Métricas de avaliação}\label{sec:metricas-avaliacao}

Para avaliação deste trabalho, foram selecionadas as seguintes métricas
que são comumente utilizadas para avaliação de segmentação de imagens:

\begin{enumerate}
\item \textit{Precision}
\item \textit{Recall}
\item \textit{F1 Score (Dice Coefficient)}
\item IoU
\end{enumerate}


\subsection{Precision}\label{sec:precision}

Para avaliação da precisão da segmentação, tem-se a seguinte definição:

\begin{equation}\label{eq:precision}
  Precision = \dfrac{TP}{TP + FP},
\end{equation}
\noindent
em que \textit{TP} são os pixels classificados corretamente e \textit{FP}
os falsos-positivos. Essa métrica penaliza a presença de falsos
positivos. Quanto menos falsos-positivos, maior será a precisão.


\subsection{Recall}\label{sec:recall}

Para avaliação do \textit{Recall} (também conhecido como Sensibilidade), tem-se a seguinte definição:

\begin{equation}\label{eq:recall}
  Recall = \dfrac{TP}{TP + FN}
\end{equation}

Nessa equação, similar à precisão, é introduzido \textit{FN} no denominador,
os falsos-negativos. Diferentemente da precisão, o \textit{recall} penaliza
a presença de falsos-negativos. Quanto menos falsos-negativos, maior
será o \textit{recall}.

\subsection{F1 Score}\label{sec:f1}

A métrica F1 Score, em segmentação de imagens também conhecida como
\textit{Dice Coefficient}, é uma soma harmônica entre \textit{precision} e
\textit{recall}. Nessa situação, tem-se a definição:


\begin{equation}\label{eq:recall}
  F1 = \dfrac{2 \cdot Precision \cdot Recall}{Precision + Recall}
\end{equation}

F1 score é uma métrica balanceada que, entre \textit{precision} e \textit{recall},
penaliza a que tiver pior avaliação.

\subsection{IoU}\label{sec:iou}

IoU~\cite{rezatofighi2019generalized}, que significa
\textit{Intersection over Union}, é uma métrica popularmente usada
para medir a precisão da segmentação de um objeto em tarefas de visão
computacional, como detecção de objetos e segmentação semântica.

A métrica IoU calcula a proporção da área de interseção entre a região
estimada e a região real (\textit{ground truth}) pela área da união dessas duas
regiões. A fórmula para calcular o IoU é:

\begin{equation}\label{eq:iou}
  IoU = \dfrac{\left| A \cap B \right|}{\left| A \cup B \right|}
\end{equation}


Nessa equação, $A$ e $B$ são matrizes de rótulos da imagem que serão
comparadas, sendo a matriz $A$ os rótulos estimados e a matriz $B$ os
reais. O valor de IoU varia de 0 a 1, com o valor 1 indicando uma
correspondência perfeita entre as regiões delimitadoras previstas e
reais, e o valor 0 indicando que não há sobreposição (pior valor). Em
geral, um IoU maior indica uma melhor precisão do modelo de
segmentação. Para ilustração, tem-se a seguinte Figura~\ref{fig:iou}:

\begin{figure}[h!]
        \captionsetup{width=10cm}
		\Caption{\label{fig:iou}
          Ilustração das métricas IoU e F1 score.
        }
		\centering
		\UFCfig{}{\fbox{\includegraphics[width=10cm]{figuras/metrics}}}{\Fonte{\citeonline{maxwell2021metrics}}}
\end{figure}
\FloatBarrier{}
