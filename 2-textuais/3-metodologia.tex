\chapter{METODOLOGIA}\label{cap:metodologia}

Para avaliação desse trabalho, foi considerado o dataset GrabCut, que
contém 50 imagens com segmentação binária e anotações parciais para
execução de segmentações assistidas, como em segmentação interativa.


\section{Métricas de avaliação}\label{sec:metricas-avaliacao}

Para avaliação desse trabalho, foi selecionado as seguintes métricas
que são comumente utilizadas para avaliação de segmentação de imagens:

\begin{enumerate}
\item IoU
\item Dice Score (F1)
\item Precision
\item Recall
\end{enumerate}

\subsection{IoU}\label{sec:iou}

IoU~\cite{rezatofighi2019generalized}, que significa
\textit{Intersection over Union}, é uma métrica popularmente usada
para medir a precisão de um objeto de segmentação em tarefas de visão
computacional, como detecção de objetos e segmentação semântica.

A métrica IoU calcula a proporção da área de interseção entre a região
estimada e a região real (ground truth) pela área da união dessas duas
regiões. A fórmula para calcular o IoU é:

\begin{equation}
  IoU = \dfrac{\left| A \cap B \right|}{\left| A \cup B \right|}
\end{equation}


Nessa equação, A e B são matrizes de rótulos da imagem onde serão
comparados, por exemplo a matriz A sendo os rótulos estimados e a
matriz B os reais. O valor de IoU varia de 0 a 1, onde 1 indica uma
correspondência perfeita entre as regiões delimitadoras previstas e
reais, e 0 indica que não há sobreposição. Em geral, um IoU maior
indica uma melhor precisão do modelo de segmentação.

A métrica mIoU é uma versão para segmentação multi-classe realizando
uma média dos IoU calculando individualmente pra cada classe,
considerando a classe negativa qualquer uma que não seja a classe alvo.
