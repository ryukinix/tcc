\chapter{Fundamentação Teórica}\label{cap:fundamentacao-teorica}

Em desenvolvimento. Fundamentação Teórica para os principais pontos da
na pesquisa:

\begin{itemize}
\item Superpixel
\item Geração de Redes Complexas
\item Matriz de Co-ocorrência
\item \gls{LCU}
\item Aprendizado Semi-supervisionado
\item Processamento de imagem
\end{itemize}

Em processamento de imagem expandir nos detalhes


\section{Aprendizado Semi-supervisionado vs. Transdução}\label{sec:teorica-aprendizado-semi-supervisionado}

A ser escrito.

\section{Aprendizado ativo}\label{sec:teorica-aprendizado-ativo}

Também conhecido como Active Learning. A ser escrito. Está relacionado
com o tópico anterior.

\section{Superpixel}\label{sec:teorica-superpixel}

A ser escrito.

\subsection{SLIC}\label{sec:teorica-superpixel-slic}

Algoritmo baseado em clusterização aleatória, um dos mais simples
algoritmos de superpixel

\section{Geração de Redes Complexas}\label{sec:teorica-redes-complexas}

A ser escrito.

\section{Matriz de Co-ocorrência}\label{sec:teorica-matriz-co-ocorrencia}

A ser escrito.


\section{Métricas de Similaridade}\label{sec:teorica-metricas-de-similaridade}

A ser escrito.


\section{LCU}\label{sec:teorica-lcu}

A ser escrito.

\section{Métricas de avaliação}

A ser escrito.

\section{EGSIS}\label{sec:teorica-egsis}

A ser escrito, algoritmo agregador que une todas as partes.
