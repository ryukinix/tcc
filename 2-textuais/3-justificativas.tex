\chapter{Justificativas}\label{cap:justificativas}

Como mencionado na introdução, os algoritmos conhecidos de segmentação
de imagens possuem restrições pertinentes que podem dificultar o uso
das técnicas em alguns problemas, como imagens médicas e edição de
imagens. Na seção \textit{revisão biblográfica} são mencionadas propriedades
pertinentes dos algoritmos propostos, como, por exemplo, possuirem complexidade
computacional linear.

Estes problemas são endereçados no desenvolvimento de uma nova técnica
de segmentação de imagens que explora outros novos ramos de
aprendizagem de máquina semi-supervisionada além das \gls{DNN}, neste
caso utilizando redes complexas e dinâmicas coletivas. Parte do
trabalho que tem sido desenvolvido em parceria com o \gls{ITA} no
projeto DNAYA\cite{DnayaMotivation}, como por exemplo a dinâmica coletiva \gls{LCU}.

Vale mencionar que, considerando a situação de pandemia que vive-se
desde 2020 com o COVID-19, a construção de tecnologias que facilitam o
diagnóstico de doenças respiratórias com auxílio de radiografias
pulmonares tem se tornado uma linha de pesquisa ainda mais relevante.
