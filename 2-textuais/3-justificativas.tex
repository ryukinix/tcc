\chapter{Justificativas}\label{cap:justificativas}

Como mencionado na introdução, os algoritmos conhecidos de segmentação
de imagens possuem restrições pertinentes que podem dificultar o uso
das técnicas em alguns problemas, como imagens médicas e edição de
imagens. Na sessão de revisão biblográfica é mencionado propriedades
pertinentes dos algoritmos propostos, que possuem complexidade
computacional linear.

Estes problemas serão endereçados no desenvolvimento de uma nova
técnica de segmentação que explora outros novos ramos de aprendizagem
de máquina semi-supervisionada além das \textit{Deep Neural Networks}
(DNN) como parte do trabalho que tem sido desenvolvido em parceria com
o Instituto Tecnológica da Aeronáutica (ITA) no projeto DNAYA
\citeonline{DnayaMotivation}, que busca criar uma biblioteca de referência
com implementação de dinâmicas coletivas e algoritmos de transformação
de grafos que tem sido desenvolvido há uma década. A utilização da
implementação desses sistemas poderá fazer parte da metodologia experimental.

Vale mencionar que considerando a situação pós-pandêmica que vive-se
em 2021 com o COVID-19, portanto a construção de tecnologias que
facilitam o diagnóstico de doenças respiratórias como o COVID-19 com
auxílio de radiografias pulmonares tem tomado interesse de institutos
de pesquisas.
