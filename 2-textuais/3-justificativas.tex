\chapter{Justificativas}\label{cap:justificativas}

Como mencionado na introdução, os algoritmos conhecidos de segmentação
de imagens possuem restrições pertinentes que podem dificultar o uso
das técnicas em alguns problemas, como imagens médicas e edição de
imagens. Na seção \textit{revisão biblográfica} são mencionadas propriedades
pertinentes dos algoritmos propostos, como por exemplo, possuirem complexidade
computacional linear.

Estes problemas serão endereçados no desenvolvimento de uma nova
técnica de segmentação que explora outros novos ramos de aprendizagem
de máquina semi-supervisionada além das \textit{Deep Neural Networks}
(\gls{DNN}) como parte do trabalho que tem sido desenvolvido em parceria com
o Instituto Tecnológico da Aeronáutica (\gls{ITA}) no projeto DNAYA
\cite{DnayaMotivation}. Ao considerar o material científico escrito no
grupo de pesquisa por cerca de 10 anos sobre dinâmicas coletivas e
redes complexas, esse projeto propõe-se criar uma biblioteca de
referência com as técnicas desenvolvidas. A utilização dessa
biblioteca fará parte da metodologia experimental.

Vale mencionar que, considerando a situação de pandemia que vive-se em
2021 com o COVID-19, a construção de tecnologias que facilitam o
diagnóstico de doenças respiratórias com auxílio de radiografias
pulmonares tem se tornado uma linha de pesquisa ainda mais relevante.
