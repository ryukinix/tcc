\chapter{RESULTADOS}\label{chap:resultados}

Ainda em desenvolvimento. O algoritmo com nome EGSIS terá sua
implementação armazenada em um projeto open-source no github. Nessa
seção conterá a comparação de variações nas técnicas propostas.

\section{Experimentos com a variação da quantidade de superpixels}\label{sec:variacao-superpixels}

Durante a execução dos experimentos, e avaliação dos resultados, foi
observado que o número de superpixels afeta diretamente a qualidade da
segmentação gerada. Entre esses experimentos, foi realizada 8
execuções com variações na técnica de extração de características
usada e o número de superpixels, sendo o número de superpixels
avaliados para 50, 100, 150 e 200.

\begin{table}[!h]
    \centering
    \Caption{\label{tab:variacao-superpixels} Resultados dos
experimentos ao variar o número de superpixels e o método de extração
de características. Em negrito os melhores resultados.}
  \begin{tabular}{lcccc}
    \toprule
    \textbf{Método}                  & \textbf{Recall} & \textbf{Precision} & \textbf{F1}     & \textbf{IoU}    \\
    \midrule \midrule
    $\gls{EGSIS}(s=50, f=gabor)$     & 0.7882          & 0.9225             & 0.8414          & 0.7412          \\
    $\gls{EGSIS}(s=100, f=gabor)$    & 0.8592          & 0.9505             & 0.8992          & 0.8232          \\
    $\gls{EGSIS}(s=150, f=gabor)$    & 0.8691          & 0.9550             & 0.9066          & 0.8354          \\
    $\gls{EGSIS}(s=200, f=gabor)$    & 0.8799          & \textbf{0.9620}    & 0.9167          & 0.8511          \\
    $\gls{EGSIS}(s=50, f=comatrix)$  & 0.7882          & 0.9225             & 0.8414          & 0.7412          \\
    $\gls{EGSIS}(s=100, f=comatrix)$ & 0.8617          & 0.9510             & 0.9010          & 0.8259          \\
    $\gls{EGSIS}(s=150, f=comatrix)$ & 0.8777          & 0.9552             & 0.9125          & 0.8441          \\
    $\gls{EGSIS}(s=200, f=comatrix)$ & \textbf{0.8877} & 0.9611             & \textbf{0.9212} & \textbf{0.8578} \\
    \bottomrule
  \end{tabular}
  \Fonte{\fonteautor}
\end{table}

Ao analisar a tabela~\ref{tab:variacao-superpixels}, é possível
perceber que o aumento de superpixels impacta positivamente na
qualidade da segmentação independente do método de extração de
características. A respeito do método de extração de características,
os resultados de ambos métodos são próximos, por outro lado, com
exceção da métrica de precisão, os melhores resultados vieram do
modelo com 200 superpixels e extração de características via matriz de
co-ocorrências.

\section{Comparação com o estado-da-arte de segmentação interativa}\label{sec:comparacao-estado-da-arte}

Nesta seção, é proposto uma avaliação de como o modelo \gls{EGSIS} se
compara ao estado-da-arte no problema de segmentação interativa.

No artigo~\cite{wang2023review}, é observado
um extenso estudo de revisão sobre os métodos de segmentação
interativa baseado em grafos, tendo como base o algoritmo GrabCut e
outros algoritmos que evoluiram nessa área de segmentação interativa
de imagens. Os autores avaliam essas técnicas na base de
dados GrabCut~\cite{rother2004grabcut} utilizando as mesmas métricas que foram
selecionadas na metodologia de avaliação desse trabalho, com exceção
do tempo de execução que por conta de diferentes tecnologias usadas
para implementação das técnicas não teríamos uma avaliação justa, que
pode ser realizada como trabalho futuro.

Esse mesmo artigo, também avalia as várias técnicas em outro dataset,
mas a rotulação inicial usada como não foi publicada, não foi possível
replicar os experimentos para o modelo \gls{EGSIS} nas mesmas
condições. Por esse motivo, a avaliação de resultados será limitada a
base de dados GrabCut.

A seguir, é possível visualizar tabelas de comparação com os
algoritmos apresentados em~\cite{wang2023review}:


\begin{table}[!h]
    \centering
    \Caption{\label{tab:resultados-estado-da-arte} Resultados
      comparativos entre o método EGSIS e métodos estado-da-arte para
      segmentação interativa baseado em grafos. Em negrito os melhores
      resultados.}
  \begin{tabular}{lcccc}
    \toprule
    \textbf{Método}                    & \textbf{Recall} & \textbf{Precision} & \textbf{F1}     & \textbf{IoU} \\
    \midrule \midrule
    GrabCut                            & 0.9668          & 0.9213             & \textbf{0.9407} & \textbf{0.8927}       \\
    LazySnapping                       & \textbf{0.9681} & 0.9104             & 0.9357          & 0.8842       \\
    OneCut                             & 0.8585          & 0.7926             & 0.7899          & 0.6974       \\
    Saliency Cuts                      & 0.8371          & 0.8892             & 0.8255          & 0.7458       \\
    Iterated Graph Cuts\footnotemark{} & 0.9614          & 0.8878             & 0.9212          & 0.8597       \\
    DenseCut                           & 0.8427          & 0.9418             & 0.8561          & 0.7927       \\
    Deep GrabCut                       & 0.8854          & 0.8774             & 0.8701          & 0.7849       \\
    $\gls{EGSIS}(s=200, f=gabor)$      & 0.8799          & \textbf{0.9620}    & 0.9167          & 0.8511       \\
    $\gls{EGSIS}(s=200, f=comatrix)$   & 0.8877          & 0.9611             & 0.9212          & 0.8578       \\
    \bottomrule
  \end{tabular}
  \Fonte{Baseado nos resultados encontrados no artigo~\cite{wang2023review}}
\end{table}


\footnotetext{Não há um nome formal pra esse método~\cite{an2013iterated}, por ora ele será chamado assim}
