Ao meu orientador Prof.\ Dr.\ Jarbas Joaci Mesquita de Sá Junior
(UFC), pela inspiração que foi pra mim, às suas inesquecíveis aulas na
UFC, além da confiança que depositou em mim e, especialmente, agradeço
as inúmeras oportunidades que me deu para recomeçar quando minha vida
pessoal se tornou um grande obstáculo.

Ao Prof.\ Dr.\ Filipe Verri (ITA) por ter contribuído fundamentalmente
na elaboração da linha de pesquisa onde este trabalho se encaixa,
pelas discussões muito inspiradoras e por suas sugestões que foram
extremamente úteis para o desenvolvimento da técnica apresentada neste
trabalho.

Aos meus professores da UFPA e UFC, que me compartilharam seus
conhecimentos e permitiram eu construir uma base de conhecimento
sólida, a qual foi muito importante para a confecção deste trabalho.

Aos meus amigos, não citarei nomes, mas que se mostraram presentes na
minha vida durante a graduação, e não apenas os que fiz na UFC, mas
incluo aqueles que cruzaram fronteiras. Para esses amigos, agradeço
especialmente pelos momentos em que estivemos conversando por anos, ao
trocar as ideias mais malucas sobre computação e os desafios nas
nossas vidas pessoais, mas também por terem participado dessa jornada
junto comigo e por me fazer lembrar de que eu não estava sozinho nos
momentos mais difíceis.

À minha família, especialmente minha mãe e minha irmã, por ter me
incentivado e apoiado na busca de uma educação de ensino superior, mesmo
com todas as dificuldades que precisei enfrentar.

Em especial, agradeço a minha esposa Taiene Francêz, por ter
acreditado que eu fosse capaz de superar todas os obstáculos
enfrentados na reta final da graduação, incluindo os mais difíceis que
envolveram a minha saúde mental.

Não menos importante, agradeço meu gato Antares pelas mordidas
inspiradoras e também pela companhia durante as madrugadas no processo
de escrita desta monografia.

Por fim, agradeço à empresa Neoway por ter me oferecido suporte financeiro e moral
durante quase toda a graduação na UFC.\@
