Ao meu orientador Prof.\ Dr.\ Jarbas Joaci Mesquita de Sá Junior
(UFC), pela inspiração que foi pra mim, às suas inesquecíveis aulas na
UFC, além da confiança que depositou em mim e, especialmente, agradeço
as inúmeras oportunidades que meu deu para recomeçar quando minha vida
pessoal se tornou um grande obstáculo.

Ao Prof.\ Dr.\ Filipe Verri (ITA) por ter contribuído fundamentalmente
na elaboração da linha de pesquisa onde este trabalho se encaixa,
pelas discussões muito inspiradoras e por suas sugestões que foram
extrememante úteis para o desenvolvimento da técnica apresentada neste
trabalho.

Aos meus professores da UFPA e UFC que me compartilharam seus
conhecimentos e permitiram eu construir uma base de conhecimento
sólida, da qual foi muito importante para a confecção deste trabalho.

Aos meus amigos que estiveram presente na minha vida durante a
graduação, e não apenas os alunos da UFC, que estivemos por anos
trocando ideias sobre computação e desafios da vida, agradeço por
terem participado dessa jornada junto comigo e também por mostrarem
que eu não estava sozinho.

À minha família, especialmente minha mãe e minha irmã, por terem me
incetivado e apoiado a busca de uma educação de ensino superior, mesmo
com todas as dificuldades que precisei enfrentar.

Em especial, agradeço a minha esposa Taiene Francêz, por ter
acreditado que eu fosse capaz de superar todas os obstáculos
enfrentados na reta final da graduação, incluindo os mais difíceis que
envolveram a minha saúde mental.

Não menos importante, agradeço meu gato Antares pelas mordidas
inspiradoras e também pela companhia durante as madrugadas no processo
de escrita desta monografia.

Por fim, agradeço à empresa Neoway por ter me oferecido suporte financeiro e moral
durante quase toda a graduação na UFC.\@
