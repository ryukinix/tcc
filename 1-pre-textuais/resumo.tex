A segmentação de imagem é uma técnica que divide a imagem em regiões
de interesse, tais como objetos em uma paisagem. Os algoritmos de
segmentação de imagem apresentam variações em seus tipos de
aprendizado, incluindo não supervisionado, supervisionado e
semi-supervisionado. No contexto de segmentação interativa, o desafio
é segmentar objetos do plano de fundo com a ajuda de rótulos iniciais
fornecidos por um usuário. Os superpixels são algoritmos de
segmentação não supervisionada utilizados como pré-segmentação para
diversos problemas de imagem, como classificação e segmentação. As
redes complexas são grafos com estruturas não triviais usadas para
representar certos domínios de dados, como regiões de uma imagem e
suas vizinhanças. A dinâmica coletiva em uma rede complexa refere-se
ao comportamento emergente e interativo de vários elementos ou atores
dentro de uma rede interconectada e complexa, onde as ações de um
elemento podem influenciar as ações dos outros. Neste trabalho,
propomos um algoritmo de segmentação de imagem semi-supervisionado que
combina as técnicas de superpixels, redes complexas e dinâmicas
coletivas. O método foi avaliado em várias condições usando a base de
dados GrabCut para segmentação interativa. Nosso método se mostrou tão
bom quanto os melhores métodos de segmentação interativa baseados em
grafos e na métrica precisão superou todos os métodos comparados.

% Separe as Keywords por ponto e vírgula
\palavraschave{redes complexas; segmentação de imagens; aprendizado semi-supervisionado; superpixel; dinamicas coletivas; segmentação interativa.}
